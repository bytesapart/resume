%-------------------------------------
% Resume Template
% Author: Osama Iqbal
% Github: https://github.com/bytesapart
% License: MIT
% -------------------------------------

% ===== Document Preamble =====
% Define a document of font 11 points, size that of US letter and of type "article" (Which is generic).
% Define a pagestyle as empty, that is no headers and footers.
% Make intercolumnar space to 0em.
\documentclass[11pt,a4paper]{article}
\pagestyle{empty}
\setlength{\tabcolsep}{0em}

% ===== Packages =====
% Use geometry package specifing letterpaper as page size and margin as 0.5 inches.
% Use the inputenc package, specifying the encoding as UTF-8
% Use the package mdwlist package for making lists and bullet points.
% Use the Lato package as font and/or use Opensans
% Use fontenc T1 so that we have 8 bit encoding for fonts, and hence, can use glyphs.
% Use textcomp package for symbols such as bullets.
% Use the fontawesome package for fontawesome.
% Use the enumitem for enumerations.
% Use the hyperref for href
\usepackage[a4paper,margin=0.5in]{geometry}
\usepackage[utf8]{inputenc}
\usepackage{mdwlist}
% \usepackage[default]{lato}
% Alternative to Lato font
\usepackage[default,scale=0.9]{opensans}
\usepackage[T1]{fontenc}
\usepackage{textcomp}
\usepackage{fontawesome}
\usepackage{enumitem}
% \usepackage{url}
\usepackage[dvipsnames]{xcolor}
\usepackage{hyperref}

% ===== Fonts to use =====
% Declare the font families to use, make an alias for FontAwesome to fontawesomeone
\DeclareFontFamily{U}{fontawesomeOne}{}
\DeclareFontShape{U}{fontawesomeOne}{m}{n}
{<-> FontAwesome--fontawesomeone}{}

% ===== Create a new command ====
% Create a command FAOne, that sets fontenconding, sets font family and selects the font, that is, fontawesome.
\DeclareRobustCommand\FAone{\fontencoding{U}\fontfamily{fontawesomeOne}\selectfont}
\makeatletter %% <- change @ so that it can be used in command sequences
\def\myrulefill#1#2{\leavevmode\leaders\hrule\@height#1\@depth#2\hfill \kern\z@}
\makeatother %% <- change @ back

% Centered w.r.t. the capital letter X
% \newcommand*\crulefill[1]{\myrulefill
%   {\dimexpr(\fontcharht\font`X+#1)/2}{\dimexpr(-\fontcharht\font`X+#1)/2}}
% Centered w.r.t. the small letter x
\newcommand*\crulefill[1]{\myrulefill{\dimexpr(1ex+#1)/2}{\dimexpr(-1ex+#1)/2}}

\definecolor{CarianBlue}{HTML}{2A3FFB}

% ===== Indent Section Style =====
% Used for sections that are not already in lists that need indentation to the level of all text in the document.
\newenvironment{indentsection}[1]
{\begin{list}{}
{\setlength{\leftmargin}{#1}}
  \item[]
}
{\end{list}}


% ===== Unindent Section Style 
% Opposite of the above. Bump a section back toward the left margin
\newenvironment{unindentsecion}[1]
{\begin{list}{}
{\setlength{\leftmargin}{-0.5#1}}
\item[]
}
{\end{list}}

% ===== Formatting two pieces of text together ====
% Format two pieces of text, one left alighned and one right alighned
\newcommand{\headerrow}[2]
{\begin{tabular*}{\linewidth}{l@{\extracolsep{\fill}}r}
#1 &
#2 \\
\end{tabular*}}

% ===== C++ specific formatting ====
% Make "C++" look pretty when used in text by touching up the plus signs.
\newcommand{\CPP}
{C\nolinebreak[4]\hspace{-0.5em}\raisebox{.22ex}{\footnotesize\bf ++}}

% ===== Resume Starts Here ====
% Actual content starts here
\begin{document}

% ===== Begin Header of the Resume =====
\begin{center}
        {\LARGE \textbf{Osama Iqbal}}\\
        Mumbai, India
        \vspace{0.05cm}
        \\
        \raisebox{-0.2\height} {\color{CarianBlue} \Large \faPhoneSquare} \ \ \href{tel:+919867143600}{+91-9867143600} \hfill\raisebox{-0.2\height}{\color{CarianBlue} \Large \faEnvelopeSquare} \ \ \href{mailto:iqbal.osama@icloud.com}{iqbal.osama@icloud.com} \hfill \raisebox{-0.2\height}{\color{CarianBlue} \Large \faGithubSquare} \ \ \href{https://github.com/bytesapart}{github.com/bytesapart} \hfill \raisebox{-0.2\height}{\color{CarianBlue} \Large \faLinkedinSquare} \ \ \href{https://linkedin.com/in/osamaiqbal}{linkedin.com/in/osamaiqbal}
\end{center}

% ===== End Header of the Resume =====

% ===== Begin Experince Section =====
\vspace{-1em}
\subsection*{\color{CarianBlue} \Large Experience \emph{(5 Years 8 Months)} \crulefill{1pt}}

\renewcommand\labelitemi{}
\renewcommand\labelitemii{$\bullet$}
\begin{itemize}[leftmargin=1em]
        \parskip=0.1em

% ===== Begin CPP, IB Experience =====
        \item
              \headerrow
              {\textbf{Canada Pension Plan, Investment Banking}}
              {\textbf{Mumbai, India}}
              \headerrow
              {\emph{Associate Software Engineer}}
              {\emph{February, 2022 -- Present}}
              \begin{itemize*}
                \item Led and Implemented OneTick Time Series Database Migration from on-prem to AWS cloud, creating and implementing custom AWS Architecture.
                \item Technologies: AWS, OneTick Timeseries, AWS DataSync, AWS Lambda, AWS EC2, AWS Aurora, AWS SNS, AWS SQS, AWS EFS, AWS Storage Gateway, C++ MFC.
              \end{itemize*}
% ===== End CPP, IB Experience =====

        
% ===== Begin Nomura Experience Section =====
        \item
              \headerrow
              {\textbf{Nomura Services India Pvt. Ltd.}}
              {\textbf{Mumbai, India}}
% ===== Nomura Associate Software Developer =====
              \headerrow
              {\emph{Associate Software Developer}}
              {\emph{May, 2020 -- February, 2022}}
              \begin{itemize*}
                \item Strengthened distributed grid computing to calculate all risks and aggregations, using spare compute cycles from more than 1000 computers in Nomura.
                \item Pioneered an easier deployment mechanism for computing grid packages to machines, saving 1 hour of support time per deployment.
                \item Technologies: C++, Boost, Clang, C\#, ASM, OpenMP, Intel Intrinsics, AVX2, Jenkins CI/CD, AWS, Docker, TDD.
              \end{itemize*}
% ===== Nomura Senior Software Development Analyst Experience =====
              \headerrow
              {\emph{Senior Software Development Analyst}}
              {\emph{October, 2018 -- May, 2020}}
              \begin{itemize*}
                \item Led end-to-end development of a business-critical engine, decoupling developers and business analysts, leading to 200+ rule change deployments per day which parses 130k+ live trades.
                \item Optimized rules engine, improving time from 15 seconds to 70 milliseconds per trade leading to decommissioning of HPC servers.
                \item Led the adoption of the rules engine for inter-system translations and communications between 4 systems.
                \item Technologies: Python, pandas, numpy, Cython, AST, Jupyter, Tibco EMS, Seaborn, Dash, VBA, TDD.
              \end{itemize*}

% ===== Nomura Software Development Analyst Experience =====
              \headerrow
              {\emph{Software Development Analyst}}
              {\emph{July, 2016 -- October, 2018}}
              \begin{itemize*}
                \item Implemented Common Risk Interchange Format’s (CRIF) Risk data aggregations for 12 legal entities, containing ~500K data points each.
                \item Strengthened quant analytical library by adding 4 different cold-start multi-processing functions.
                \item Technologies: Python, PySpark, MapReduce, Hadoop, Dask, Flask, Django, multiproc and multithread libraries, TDD.
              \end{itemize*}

% ===== Nomura Software Development Intern Experience =====
              \headerrow
              {\emph{Software Development Intern}}
              {\emph{January, 2016 -- July, 2016}}
              \begin{itemize*}
                \item Conceived a GUI tool that makes FIX 4.4 protocol messages, sending 10k messages to an Exchange Simulator.
                \item Created a client-server socket-based distributed test-case runner configured on spare machines - running 7k test cases simultaneously.
                \item Technologies: Python, QuickFix, win32com, websocket library, PHP, Bootstrap, Javascript, jQuery.

              \end{itemize*}

% ===== End Nomura Experience =====

\end{itemize}
% ===== End Experience Section =====

% ===== Begin Education Section =====
% \pagebreak
% \hrule
\vspace{-1em}
\subsection*{\color{CarianBlue} \Large Education \crulefill{1pt}} 

\begin{itemize}[leftmargin=1em,noitemsep]
    \parskip=0.1em

      \item
            \headerrow
            {\textbf{WorldQuant University}}
            {\textbf{Mumbai, India}}
            \headerrow
            {\emph{M.S. Quantitative Finance} \textbf{(80.0\%/100.0\%)}}
            {\emph{January, 2017 -- January, 2019}}

            \headerrow
            {\textbf{Sardar Patel Institute of Technology, Mumbai University}}
            {\textbf{Mumbai, India}}
            \headerrow
            {\emph{Master of Computer Application} \textbf{(GPA: 8.29/10.0)}}
            {\emph{July, 2014 -- July, 2016}}

            \headerrow
            {\textbf{Valia C.L. College of Science, Mumbai University}}
            {\textbf{Mumbai, India}}
            \headerrow
            {\emph{Bachelor of Science (Information Technology)} \textbf{(82.5/100.0)}}
            {\emph{July, 2011 -- March, 2014}}
\end{itemize}

% ===== End Education Section =====

% ===== Begin Skills Section =====
\vspace{-1em}
\subsection*{\color{CarianBlue} \Large Skills and Certifications \crulefill{1pt}}

\hyphenpenalty=1000
\begin{itemize}[leftmargin=1em,noitemsep]
% \begin{itemize}[leftmargin=1em]
        \item \textbf{Languages:}
              Python, C++, C\#, SQL
        \item \textbf{Technologies:}
              Pandas, Numpy, Dask, PySpark, Hadoop, Hive, MapReduce, Seaborn, Flask, Django, Cython, Boost, OpenMP, ASM, Intel Intrinsics, AVX2, Tibco EMS, Git, Github-Actions, Docker, Jupyter, Jenkins CI/CD, CMake, TDD.
        % \item \textbf{Working Knowledge:}
        %       PyTorch, Tensorflow, JAX, Reinforcement Learning, Deep Learning, Machine Learning, CUDA, C\#,Apache Kafka, Typescript, Javascipt, ReactJS, React Native, RESTful services, Microservices, ETL, Databricks, vtk.js, itk.js, Groovy, ElasticSearch, Logstash, Kibana, Bazel.
        \item \textbf{Certifications: } AWS Certified Developer Associate, Datacamp Data Scientist with Python.
\end{itemize}

% ===== End Skills Section =====

% ===== Begin Certification Section =====
% \hrule
% \vspace{-1em}
% \subsection*{\Large Certifications}
% \begin{itemize}[leftmargin=1em,noitemsep]
%         \item \textbf{AWS:} Certified Developer Associate
%         \item \textbf{Datacamp:} Data Scientist (Python)
        % \item \textbf{Coursera: } Machine Learning
        % \item \textbf{Deeplearning.ai: } Deeplearning Specialisation
        % \item \textbf{Deeplearning.ai:} MLOps for Production
% \end{itemize}
% \hrule
% ===== End Certification Section =====


\end{document}
